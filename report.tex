% Setting up the document class
\documentclass[a4paper,12pt]{article}

% Including necessary packages
\usepackage{geometry}
\usepackage{amsmath}
\usepackage{booktabs}
\usepackage{enumitem}
\usepackage{hyperref}
\usepackage{parskip}

% Configuring page geometry
\geometry{margin=1in}

% Setting up hyperlinks
\hypersetup{
    colorlinks=true,
    linkcolor=blue,
    urlcolor=blue
}

% Setting font to DejaVu Sans
\usepackage{dejavu}

% Beginning the document
\begin{document}

% Creating the title
\title{Breast Cancer Classification and Prediction System: Technical Report}
\author{}
\date{May 2025}
\maketitle

% Adding an abstract
\begin{abstract}
This report summarizes the development and performance of machine learning models for breast cancer classification and a web-based prediction system using the scikit-learn breast cancer dataset. It covers classification results, the Random Forest-based prediction system, and key observations.
\end{abstract}

% Section for Classification Observations
\section{Classification Observations}
\subsection{Wine Dataset}
\begin{itemize}[noitemsep]
    \item \textbf{Logistic Regression}: Highest accuracy (\textasciitilde0.9815) due to linear separability of features.
    \item \textbf{SVM}: Lower accuracy (\textasciitilde0.7593) with default parameters; requires kernel tuning.
    \item \textbf{Decision Tree}: Good performance with \texttt{random\_state=42} (\textasciitilde0.9630); variable without (\textasciitilde0.9444).
\end{itemize}

\subsection{Breast Cancer Dataset}
\begin{itemize}[noitemsep]
    \item \textbf{Random Forest}: Best performer (\textasciitilde0.9591), leveraging ensemble learning.
    \item \textbf{Logistic Regression}: Strong accuracy (\textasciitilde0.9474), suited to dataset’s structure.
    \item \textbf{SVM}: Moderate accuracy (\textasciitilde0.9123); needs optimization.
    \item \textbf{Decision Tree}: Stable with \texttt{random_state=42} (\textasciitilde0.9181); variable otherwise (\textasciitilde0.9064).
\end{itemize}

\begin{table}[h]
    \centering
    \caption{Classification Performance}
    \begin{tabular}{lcc}
        \toprule
        \textbf{Model} & \textbf{Wine Accuracy} & \textbf{Breast Cancer Accuracy} \\
        \midrule
        Logistic Regression & 0.9815 & 0.9474 \\
        SVM & 0.7593 & 0.9123 \\
        Decision Tree (fixed) & 0.9630 & 0.9181 \\
        Decision Tree (variable) & 0.9444 & 0.9064 \\
        Random Forest & -- & 0.9591 \\
        \bottomrule
    \end{tabular}
\end{table}

% Section for Prediction System Observations
\section{Prediction System Observations}
\subsection{System Overview}
\begin{itemize}[noitemsep]
    \item \textbf{Model}: Random Forest Classifier (\textasciitilde0.9591 accuracy), integrated into Flask web app.
    \item \textbf{Input}: Top 5 features selected via feature importance (e.g., worst perimeter, mean concave points).
    \item \textbf{Output}: Predicts Malignant (0) or Benign (1) with confidence score.
\end{itemize}

\subsection{Design and Functionality}
\begin{itemize}[noitemsep]
    \item \textbf{Frontend}: Responsive form in home page, styled with Tailwind CSS, validated via JavaScript.
    \item \textbf{Backend}: Flask processes inputs, scales features, and serves predictions.
    \item \textbf{Feature Importance}: Reduced input from 30 to 5 features, balancing simplicity and accuracy.
\end{itemize}

\subsection{Performance and Limitations}
\begin{itemize}[noitemsep]
    \item \textbf{Performance}: Fast, reliable predictions; highly user-friendly with minimal inputs.
    \item \textbf{Limitations}: Requires accurate feature inputs; no guidance on feature ranges.
    \item \textbf{Future Work}: Add feature range validation, visualize feature importance.
\end{itemize}

% Ending the document
\end{document}